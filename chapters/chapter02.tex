\section{一些插入功能}
\subsection{插入公式}
行内公式$v-\varepsilon+\phi=2$。

插入行间公式如\autoref{Euler}:
\begin{equation}
    v-\varepsilon+\phi=2
    \label{Euler}
\end{equation}

\subsection{插入图片}
SWJTU校徽如\autoref{SWJTU}所示,注意这里使用了\verb|~\autoref{}|命令,也就是会自动生成“图”“式”等前缀,无需手动输入。

\begin{figure}[!htbp]
    \centering
    \includegraphics[width =0.4\textwidth]{figures/swjtu_logo2.pdf}
    \caption{西南交通大学}
    \label{SWJTU}
\end{figure}

插入上面图片的代码:

\begin{verbatim}
    \begin{figure}[!htbp]
        \centering
        \includegraphics[width =0.4\textwidth]{figures/ucas_logo.pdf}
        \caption{西南交通大学}
        \label{SWJTU}
    \end{figure}
\end{verbatim}

\subsection{插入文本框}
本模板定义了一个圆角灰底的文本框,使用简化命令\verb|\tbox{}|即可,如果你不喜欢,可以前往 \texttt{swjtuReport.sty}对其进行修改。

\tbox{
    这是一个圆角灰底的文本框
}

\subsection{插入表格}
本模板文件如\autoref{doc}所示。
\begin{table}[!htbp]
    \centering
    \begin{tabular}{l|l}
    \hline
        文件名 & 说明 \\
        \hline
        \texttt{main.tex}  & 主文件 \\
        \texttt{references.bib} & 参考文献 \\
        \texttt{swjtuReport.sty}  & 文档格式控制\\
        \texttt{figures}  & 图片文件夹 \\
        \hline
    \end{tabular}
    \caption{本模板文件组成}
    \label{doc}
\end{table}

\subsection{插入数学逻辑环境}
\begin{Theorem}   % 定理
\end{Theorem}

\begin{Lemma}   % 引理
\end{Lemma}

\begin{Corollary}   % 推论
\end{Corollary}

\begin{Proposition}   % 命题
\end{Proposition}

\begin{Definition}   % 定义
\end{Definition}

\begin{Example}   % 例
\end{Example}

\begin{proof}   %证明
\end{proof}

\subsection{插入参考文献}
直接使用\verb|\cite{}|即可。

例如:

   \textit{ 此处引用了文献\cite{0Isaac}。此处引用了文献\cite{2016The}}

引用过的文献会自动出现在参考文献中。
